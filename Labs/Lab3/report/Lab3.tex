\documentclass[12pt,a4paper]{article}

%----------------------------------------------------------------------------------------
% PACKAGES
%----------------------------------------------------------------------------------------
\usepackage[a4paper,margin=25mm]{geometry}
\usepackage{fontspec}
\setmainfont{Times New Roman}

\usepackage{graphicx,float,booktabs,array,multirow,url}
\usepackage{gensymb}
\usepackage[british]{babel}
\usepackage[square,numbers,sort&compress]{natbib}
\usepackage{caption,subcaption}
\usepackage{pgfplots}
\pgfplotsset{compat=1.9}

\usepackage{minted}

\usepackage{fancyhdr}
\setlength{\headheight}{15pt}
\addtolength{\topmargin}{-2.5pt}

\usepackage{amssymb}
\usepackage{amsmath,amssymb,amsfonts}

\usepackage{silence}
\WarningFilter{gensymb}{Not defining}

\usepackage[colorlinks=true,
            linkcolor=blue,
            citecolor=blue,
            urlcolor=blue]{hyperref}

%----------------------------------------------------------------------------------------
% CUSTOM ABSTRACT FORMAT
%----------------------------------------------------------------------------------------
\makeatletter
\renewenvironment{abstract}{
    \begin{center}
        \large\bfseries \abstractname
    \end{center}
    \begin{quote}\small
}{
    \end{quote}
}
\makeatother

%----------------------------------------------------------------------------------------
% USER INFORMATION
%----------------------------------------------------------------------------------------
\newcommand{\studentname}{Yuwei ZHAO}
\newcommand{\studentnumber}{23020036096}
\newcommand{\labgroup}{Group \#31}
\newcommand{\labdate}{2025-11-20}
\newcommand{\course}{Robotics Integration Group Project I}
\newcommand{\labtitle}{Lab 3 Report}

%----------------------------------------------------------------------------------------
% TITLE
%----------------------------------------------------------------------------------------
\title{
    \vspace{-1cm}
    \textbf{\labtitle}\\[0.3em]
    \Large \course
}
\author{
    \studentname~(\studentnumber)\\
    \labgroup \quad \labdate
}
\date{}

%----------------------------------------------------------------------------------------
% PAGE STYLE
%----------------------------------------------------------------------------------------
\pagestyle{fancy}
\fancyhf{}
\fancyhead[L]{\labtitle}
\fancyhead[C]{\studentname~(\studentnumber)}
\fancyhead[R]{\labgroup}
\fancyfoot[C]{\thepage}
\renewcommand{\headrulewidth}{0.4pt}
\renewcommand{\footrulewidth}{0pt}

\setlength{\parindent}{0em}
\setlength{\parskip}{0.75em}

%----------------------------------------------------------------------------------------
% DOCUMENT BODY
%----------------------------------------------------------------------------------------
\begin{document}

\maketitle

%----------------------------------------------------------------------------------------
\begin{abstract}

See Resources on \href{https://github.com/RamessesN/Robotics_MIT}{github.com/RamessesN/Robotics\_MIT}.
\end{abstract}

%----------------------------------------------------------------------------------------
\section{Introduction}

%----------------------------------------------------------------------------------------
\section{Procedure}
%----------------------------------------------------------------------------------------
\subsection{Individual Work}
\subsubsection{Transformations in Practice}
\begin{enumerate}
    \item \textbf{MESSAGE VS. TF}
    \begin{itemize}
        \item \textbf{Assume we have an incoming \textit{geometry\_msgs::Quaternion quat\_msg} that holds the pose of our robot. 
        We need to save it in an already defined \textit{tf2::Quaternion quat\_tf} for further calculations. 
        Write one line of C++ code to accomplish this task.} \\ [0.5em]
        \textbf{Solution}:
        To convert a \textit{geometry\_msgs::Quaternion} into a \textit{tf2::Quaternion}, simply initialize the latter with the x, y, z, w components of the incoming message:
        \begin{minted}[fontsize=\footnotesize,breaklines]{cpp}
quat_tf = tf2::Quaternion(quat_msg.x, quat_msg.y, quat_msg.z, quat_msg.w);
        \end{minted}
        \begin{figure}[H]
            \centering
            \includegraphics[width=0.6\textwidth]{source/img/tf2_Quaternion.png}
            \caption{\textit{tf2 Quaternion doc}}
            \label{fig:tf2_quaternion_doc}
        \end{figure}

        \item \textbf{Assume we have just estimated our robot’s newest rotation and it’s saved in a variable called \textit{quat\_tf} of type \textit{tf2::Quaternion}. 
        Write one line of C++ code to convert it to a \textit{geometry\_msgs::Quaternion} type. 
        Use \textit{quat\_msg} as the name of the new variable.} \\ [0.5em]
        \textbf{Solution}:
        \begin{minted}[fontsize=\footnotesize,breaklines]{cpp}
geometry_msgs::Quaternion quat_msg = tf2::toMsg(quat_tf);
        \end{minted}

        \item If you just want to know the scalar value of a \textit{tf2::Quaternion}, what member function will you use? \\ [0.5em]
    \end{itemize}

    \item \textbf{CONVERSION}
    \begin{itemize}
        \item Assume you have a \textit{tf2::Quaternion quat\_t}. 
        How to extract the yaw component of the rotation with just one function call?

        \item Assume you have a \textit{geometry\_msgs::Quaternion quat\_msg}. How to you convert it to an Eigen 3-by-3 matrix? 
        Refer to \href{https://docs.ros.org/en/jade/api/tf2_eigen/html/index.html}{this} for possible functions. You probably need two function calls for this.
    \end{itemize}
\end{enumerate}

\subsubsection{Modelling and control of UAVs}
\begin{enumerate}
    \item STRUCTURE OF QUADROTORS
    \item CONTROL OF QUADROTORS
\end{enumerate}

%----------------------------------------------------------------------------------------
\subsection{Team Work}
\subsubsection{Trajectory tracking for UAVs}
\subsubsection{Launching the TESSE simulator with ROS bridge}
\subsubsection{Implement the controller}
\subsubsection{Simulator conventions}
\subsubsection{Geometric controller for the UAV}

%----------------------------------------------------------------------------------------
\section{Reflection and Analysis}

%----------------------------------------------------------------------------------------
\section{Conclusion}

\newpage

%----------------------------------------------------------------------------------------
% Source Code
%----------------------------------------------------------------------------------------
\pagestyle{empty}
\section{Source Code}
\label{section:source_code}
\begin{itemize}
    \item 

    \item 
\end{itemize}



\end{document}