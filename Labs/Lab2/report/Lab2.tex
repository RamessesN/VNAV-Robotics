\documentclass[12pt,a4paper]{article}

%----------------------------------------------------------------------------------------
% PACKAGES
%----------------------------------------------------------------------------------------
\usepackage[a4paper,margin=25mm]{geometry}
\usepackage{fontspec}
\setmainfont{Times New Roman}

\usepackage{graphicx,float,booktabs,array,multirow,url}
\usepackage{gensymb}
\usepackage[british]{babel}
\usepackage[square,numbers,sort&compress]{natbib}
\usepackage{caption,subcaption}
\usepackage{pgfplots}
\pgfplotsset{compat=1.9}

\usepackage{minted}

\usepackage{fancyhdr}
\setlength{\headheight}{15pt}
\addtolength{\topmargin}{-2.5pt}

\usepackage{silence}
\WarningFilter{gensymb}{Not defining}

\usepackage[colorlinks=true,
            linkcolor=blue,
            citecolor=blue,
            urlcolor=blue]{hyperref}

%----------------------------------------------------------------------------------------
% CUSTOM ABSTRACT FORMAT
%----------------------------------------------------------------------------------------
\makeatletter
\renewenvironment{abstract}{
    \begin{center}
        \large\bfseries \abstractname
    \end{center}
    \begin{quote}\small
}{
    \end{quote}
}
\makeatother

%----------------------------------------------------------------------------------------
% USER INFORMATION
%----------------------------------------------------------------------------------------
\newcommand{\studentname}{Yuwei ZHAO}
\newcommand{\studentnumber}{23020036096}
\newcommand{\labgroup}{Group \#31}
\newcommand{\labdate}{2025-11-12}
\newcommand{\course}{Robotics Integration Group Project I}
\newcommand{\labtitle}{Lab 2 Report}

%----------------------------------------------------------------------------------------
% TITLE
%----------------------------------------------------------------------------------------
\title{
    \vspace{-1cm}
    \textbf{\labtitle}\\[0.3em]
    \Large \course
}
\author{
    \studentname~(\studentnumber)\\
    \labgroup \quad \labdate
}
\date{}

%----------------------------------------------------------------------------------------
% PAGE STYLE
%----------------------------------------------------------------------------------------
\pagestyle{fancy}
\fancyhf{}
\fancyhead[L]{\labtitle}
\fancyhead[C]{\studentname~(\studentnumber)}
\fancyhead[R]{\labgroup}
\fancyfoot[C]{\thepage}
\renewcommand{\headrulewidth}{0.4pt}
\renewcommand{\footrulewidth}{0pt}

\setlength{\parindent}{0em}
\setlength{\parskip}{0.75em}

%----------------------------------------------------------------------------------------
% DOCUMENT BODY
%----------------------------------------------------------------------------------------
\begin{document}

\maketitle

%----------------------------------------------------------------------------------------
\begin{abstract}
This report presents the setup and experimentation process of Lab 2. 
The primary objective of this lab is to install and configure the Robot Operating System (\textit{ROS}) environment, establish a working ROS workspace and gain hands-on experience with ROS nodes, topics, and message communication. 
Through a series of practical exercises, the lab introduces core \textit{ROS} concepts such as node creation, publisher–subscriber mechanisms, and visualization using \textit{rqt\_graph} and \textit{roscore}. 
The experiment provides a foundational understanding of how \textit{ROS} enables modular and distributed robotics software development.

See Resources on \href{https://github.com/RamessesN/Robotics_MIT}{git@github.com:RamessesN/Robotics\_MIT}.
\end{abstract}

%----------------------------------------------------------------------------------------
\section{Introduction}
This laboratory session focuses on the installation, configuration and initial exploration of the Robot Operating System (\textit{ROS}), which serves as the middleware framework for subsequent robotics development. 
Before implementing perception, planning or control modules, it is essential to understand the \textit{ROS} architecture and its communication mechanisms. 
The experiment involves setting up the \textit{ROS} environment, creating and managing a catkin workspace, and developing simple publisher and subscriber nodes to exchange data through topics. 

%----------------------------------------------------------------------------------------
\section{Procedure}
%----------------------------------------------------------------------------------------
\subsection{Part I}
\subsubsection{Objective}
To set up the \textit{ROS} environment on the \textit{Ubuntu 24.04} operating system, including the installation of essential packages and the configuration of a functional Catkin workspace.

\subsubsection{Methodology}
The initial approach for installing \textit{ROS Noetic} was as follows:
\begin{enumerate}
    \item Add the \textit{ROS} package sources and install \textit{ROS Noetic}:
    \begin{minted}[fontsize=\footnotesize,breaklines]{bash}
sudo sh -c 'echo "deb http://packages.ros.org/ros/ubuntu $(lsb_release -sc) main" > /etc/apt/sources.list.d/ros-latest.list'
curl -s https://raw.githubusercontent.com/ros/rosdistro/master/ros.asc | sudo apt-key add -
sudo apt update
sudo apt install ros-noetic-desktop-full
    \end{minted}

    \item Configure the \textit{ROS} environment and install additional dependencies:
    \begin{minted}[fontsize=\footnotesize,breaklines]{bash}
echo "source /opt/ros/noetic/setup.bash" >> ~/.bashrc
source ~/.bashrc
sudo apt install python3-rosdep python3-rosinstall python3-rosinstall-generator \
python3-vcstool build-essential python3-catkin-tools python-is-python3
    \end{minted}

    \item Initialize \textit{rosdep} to enable dependency management:
    \begin{minted}[fontsize=\footnotesize,breaklines]{bash}
sudo rosdep init
rosdep update
    \end{minted}
\end{enumerate}

However, the above method is not fully compatible with \textit{Ubuntu 24.04}.  
To address this, an alternative installation procedure using the \textit{Shrike} repository was adopted, which provided a successful setup.

\begin{enumerate}
    \item Clone the \href{https://github.com/Minoic-Intelligence/shrike.git}{\textit{Shrike}} repository from GitHub:
    \begin{minted}[fontsize=\footnotesize,breaklines]{bash}
git clone git@github.com:Minoic-Intelligence/shrike.git
    \end{minted}

    \item Follow the instructions provided in the \textit{Shrike} repository to complete the installation.  
    The resulting terminal output is shown in Figure~\ref{fig:ros_noetic_installation}.
    \begin{minted}[fontsize=\footnotesize,breaklines]{bash}
./scripts/install_ubuntu24.sh
./src/catkin/bin/catkin_make_isolated --install -DCMAKE_BUILD_TYPE=Release
source ./install_isolated/setup.bash
    \end{minted}
\end{enumerate}

\subsubsection{Observations}
\begin{figure}[H]
    \centering
    \begin{minipage}[b]{0.48\textwidth}
        \includegraphics[width=\textwidth]{./source/img/partI_1.png}
    \end{minipage}
    \hfill
    \begin{minipage}[b]{0.48\textwidth}
        \includegraphics[width=\textwidth]{./source/img/partI_2.png}
    \end{minipage}
    \caption{Installation of \textit{ROS Noetic} on \textit{Ubuntu 24.04}}
    \label{fig:ros_noetic_installation}
\end{figure}

\subsubsection{Discussion}
Installing \textit{ROS Noetic} on \textit{Ubuntu 24.04} presents several challenges due to incompatibilities between the standard Noetic packages and the latest system libraries.  
By utilizing the \textit{Shrike} repository—which provides customized build scripts tailored for Ubuntu 24.04—the installation process was completed successfully.  
This outcome highlights the necessity of adapting traditional installation procedures to evolving system environments and demonstrates the value of community-maintained solutions when official support is unavailable.

%----------------------------------------------------------------------------------------
\subsection{Part II}
\subsubsection{Objective}
To deepen the understanding of the \textit{ROS} communication framework by implementing publisher and subscriber nodes, exploring topic-based message exchange, and visualizing the inter-node topology. 
This experiment establishes the foundation for modular, event-driven robotic behavior through:
\begin{enumerate}
    \item Implementing \textit{ROS} nodes that publish and subscribe to custom message types.
    \item Demonstrating data flow between nodes via topics and verifying correct message delivery.
    \item Utilizing diagnostic and visualization tools (such as {\tt rqt\_graph}) to examine node–topic connectivity.
    \item Ensuring the workspace is properly configured to build, launch, and manage multiple nodes within a \textit{ROS} ecosystem.
\end{enumerate}

\subsubsection{Methodology}
\begin{itemize}
    \item \textbf{\textit{ROS} Master} \\[0.2em]
    Start the \textit{ROS} Master using the following command.  
    The resulting output is shown in Figure~\ref{fig:ros_master}.
    \begin{minted}[fontsize=\footnotesize,breaklines]{bash}
roscore
    \end{minted}

    \item \textbf{\textit{ROS} Nodes}
    \begin{enumerate}
        \item Launch the \textit{turtlesim} node in a new terminal:
        \begin{minted}[fontsize=\footnotesize,breaklines]{bash}
rosrun turtlesim turtlesim_node
        \end{minted}

        \item List active nodes using:
        \begin{minted}[fontsize=\footnotesize,breaklines]{bash}
rosnode list
        \end{minted}

        \item Launch the \textit{turtle\_teleop\_key} node to control the turtle with keyboard input:
        \begin{minted}[fontsize=\footnotesize,breaklines]{bash}
rosrun turtlesim turtle_teleop_key
        \end{minted}

        \item The corresponding runtime output is illustrated in Figure~\ref{fig:ros_nodes}.
    \end{enumerate}

    \item \textbf{\textit{ROS} Topics}
    \begin{enumerate}
        \item Visualize the communication graph of running nodes and topics:
        \begin{minted}[fontsize=\footnotesize,breaklines]{bash}
rosrun rqt_graph rqt_graph
        \end{minted}

        \item Monitor velocity commands sent to the turtle in real time:
        \begin{minted}[fontsize=\footnotesize,breaklines]{bash}
rostopic echo /turtle1/cmd_vel
        \end{minted}

        \item Create a simple \textit{C++} example node:
        \begin{minted}[fontsize=\footnotesize,breaklines,linenos,frame=single]{cpp}
#include <ros/ros.h>

int main(int argc, char** argv) {
    ros::init(argc, argv, "example_node");
    ros::NodeHandle n;

    ros::Rate loop_rate(50);

    while (ros::ok()) {
        ros::spinOnce();
        loop_rate.sleep();
    }
    return 0;
}
        \end{minted}

        To compile and execute this example using \textit{Shrike}, follow these steps:
        \begin{minted}[fontsize=\footnotesize,breaklines]{bash}
cd ~/shrike/ros_ws/src
catkin_create_pkg ros_sample roscpp std_msgs
        \end{minted}

        The resulting workspace structure is:
        \begin{minted}[fontsize=\footnotesize,breaklines]{bash}
- ros_ws/
  - src/
    - ros_sample/
      - CMakeLists.txt
      - include/
      - package.xml
      - src/
        - ros_sample.cpp
        \end{minted}

        Replace \textit{example\_node.cpp} with \textit{ros\_sample.cpp} and update the \textit{CMakeLists.txt} file as follows:
        \begin{minted}[fontsize=\footnotesize,breaklines,frame=single]{cpp}
cmake_minimum_required(VERSION 3.0.2)
project(ros_sample)

find_package(catkin REQUIRED COMPONENTS
  roscpp
  std_msgs
)

catkin_package()

include_directories(
  ${catkin_INCLUDE_DIRS}
)

add_executable(ros_sample_node src/ros_sample.cpp)
target_link_libraries(ros_sample_node ${catkin_LIBRARIES})
        \end{minted}

        Build and source the workspace:
        \begin{minted}[fontsize=\footnotesize,breaklines]{bash}
cd ~/shrike/ros_ws
catkin_make_isolated --install -DCMAKE_BUILD_TYPE=Release
source ~/shrike/ros_ws/devel_isolated/setup.bash
rosrun ros_sample ros_sample_node
        \end{minted}

        Verify that the node is active using \texttt{rosnode list}.  
        The corresponding results are shown in Figure~\ref{fig:ros_topics}.

        \item \textbf{\textit{TF} Tools}
        \begin{itemize}
            \item Launch the Turtle TF demo:
            \begin{minted}[fontsize=\footnotesize,breaklines]{bash}
roslaunch turtle_tf turtle_tf_demo.launch
            \end{minted}

            \textit{Note:} On Ubuntu 24.04 with \textit{Shrike}, older Noetic packages may depend on Python 2, while only Python 3 is installed.  
            To resolve this compatibility issue, create a symbolic link:
            \begin{minted}[fontsize=\footnotesize,breaklines]{bash}
sudo ln -s /usr/bin/python3 /usr/bin/python
            \end{minted}

            \item Visualize the TF tree using:
            \begin{minted}[fontsize=\footnotesize,breaklines]{bash}
rosrun rqt_tf_tree rqt_tf_tree
            \end{minted}
            The visualization shows three coordinate frames: \textit{world} (parent), \textit{turtle1}, and \textit{turtle2}.

            \item Inspect transformations in real time:
            \begin{minted}[fontsize=\footnotesize,breaklines]{bash}
rosrun tf tf_echo /turtle1 /turtle2
            \end{minted}

            \item The resulting output is illustrated in Figure~\ref{fig:tf_tools}.
        \end{itemize}

        \item \textbf{Using \textit{RViz}} \\[0.2em]
        Run the following command to launch \textit{RViz}:
        \begin{minted}[fontsize=\footnotesize,breaklines]{bash}
LIBGL_ALWAYS_SOFTWARE=1 rviz
        \end{minted}
        Software rendering is required because \textit{RViz} relies on OpenGL, which may not be fully supported in virtualized environments.  
        The visualization result is shown in Figure~\ref{fig:rviz}.
    \end{enumerate}

    \textit{Notice:} Modify the \textit{view\_frames} script for Python 3.x compatibility:
    \begin{minted}[fontsize=\footnotesize,breaklines,frame=single]{python}
try:
    vstr = subprocess.Popen(args, stdout=subprocess.PIPE,
                            stderr=subprocess.STDOUT).communicate()[0]
    vstr = vstr.decode('utf-8')
except OSError as ex:
    print("Warning: Could not execute `dot -V`. Is graphviz installed?")
    sys.exit(-1)

v = distutils.version.StrictVersion('2.16')
r = re.compile(r".*version (\d+\.?\d*)")
print(vstr)
m = r.search(vstr)
    \end{minted}
\end{itemize}

\subsubsection{Observations}
\begin{itemize}
    \item \textit{ROS} Master
    \begin{figure}[H]
        \centering
        \includegraphics[width=0.6\textwidth]{./source/img/partII_1.png}
        \caption{ROS Master}
        \label{fig:ros_master}
    \end{figure}

    \item \textit{ROS} Nodes
    \begin{figure}[H]
        \centering
        \begin{minipage}[b]{\textwidth}
            \begin{minipage}[b]{0.35\textheight}
                \includegraphics[width=\textwidth]{./source/img/partII_2.png}
                \caption*{Turtlesim node running}
            \end{minipage}
            \hfill
            \begin{minipage}[b]{0.25\textheight}
                \includegraphics[width=\textwidth]{./source/img/partII_3.png}
                \caption*{rosnode list output}
            \end{minipage}
        \end{minipage}
        \vfill
        \begin{minipage}[b]{\textwidth}
            \begin{minipage}[b]{0.35\textheight}
                \includegraphics[width=\textwidth]{./source/img/partII_4.png}
                \caption*{Turtle teleop node running}
            \end{minipage}
            \hfill
            \begin{minipage}[b]{0.25\textheight}
                \includegraphics[width=\textwidth]{./source/img/partII_5.png}
                \caption*{rosnode list output}
            \end{minipage}
        \end{minipage}
        \caption{ROS nodes}
        \label{fig:ros_nodes}
    \end{figure}

    \item \textit{ROS} Topics
    \begin{figure}[H]
        \centering
        \begin{minipage}[b]{\textwidth}
            \centering
            \begin{minipage}[b]{0.25\textheight}
                \centering
                \includegraphics[width=\textwidth]{./source/img/partIII_1.png}
                \caption*{rqt\_graph output}
            \end{minipage}
            \hfill
            \begin{minipage}[b]{0.35\textheight}
                \centering
                \includegraphics[width=\textwidth]{./source/img/partIII_2.png}
                \caption*{Velocity topic output}
            \end{minipage}
        \end{minipage}
        \vfill
        \begin{minipage}[b]{\textwidth}
            \centering
            \begin{minipage}[b]{0.48\textwidth}
                \centering
                \includegraphics[width=\textwidth]{./source/img/partIII_3.png}
                \caption*{ROS node running}
            \end{minipage}
            \hfill
            \begin{minipage}[b]{0.48\textwidth}
                \centering
                \includegraphics[width=\textwidth]{./source/img/partIII_4.png}
                \caption*{rosnode list output}
            \end{minipage}
        \end{minipage}
        \vfill
        \begin{minipage}[b]{\textwidth}
            \centering
            \includegraphics[width=0.6\textwidth]{./source/img/partIII_5.png}
            \caption*{Roslaunch output}
        \end{minipage}
        \caption{ROS topics}
        \label{fig:ros_topics}
    \end{figure}

    \begin{figure}[H]
        \centering
        \begin{minipage}[b]{0.48\textwidth}
            \centering
            \includegraphics[width=\textwidth]{./source/img/partIII_6.png}
            \caption*{(a) rqt\_tf\_tree output}
        \end{minipage}
        \hfill
        \begin{minipage}[b]{0.48\textwidth}
            \centering
            \includegraphics[width=\textwidth]{./source/img/partIII_7.png}
            \caption*{(b) tf\_echo output}
        \end{minipage}
        \caption{TF tools}
        \label{fig:tf_tools}
    \end{figure}

    \begin{figure}[H]
        \centering
        \includegraphics[width=0.6\textwidth]{./source/img/partIII_8.png}
        \caption{RViz}
        \label{fig:rviz}
    \end{figure}
\end{itemize}

\subsubsection{Discussion}
This section demonstrates the core principles of inter-node communication and system visualization within the \textit{ROS} framework.  
By implementing publisher and subscriber nodes and observing message flow using \textit{rqt\_graph} and \textit{rostopic echo}, it becomes clear how topic-based communication enables modular and decoupled node interaction.  
The ability to monitor live message streams allows developers to validate correct behavior and quickly identify configuration or runtime issues.

The use of \textit{TF} frames further illustrates the importance of coordinate transformation management in robotic systems.  
The hierarchical relationship among the \textit{world}, \textit{turtle1}, and \textit{turtle2} frames highlights how spatial relationships are maintained and broadcast across the \textit{ROS} network.  
Tools such as \textit{rqt\_tf\_tree} and \textit{tf\_echo} provide intuitive means of visualizing and debugging frame hierarchies, reinforcing understanding of kinematic relationships and transformation propagation.

Running \textit{RViz} in a virtualized environment emphasizes practical considerations for deploying visualization tools.  
The need to enforce software rendering with \texttt{LIBGL\_ALWAYS\_SOFTWARE=1} underscores how hardware acceleration and OpenGL support directly affect performance and usability.

%----------------------------------------------------------------------------------------
\subsection{Part III}
\subsubsection{Objective}
To utilize \textit{TF} to control the movement of drones.

\subsubsection{Methodology}
\begin{enumerate}
    \item \textbf{Setting up the workspace} \\
    First, create a Catkin workspace named \textit{vnav\_ws} under the home directory:
    \begin{minted}[fontsize=\footnotesize,breaklines]{bash}
mkdir -p ~/vnav_ws/src
cd ~/vnav_ws/
catkin init
    \end{minted}

    The output should appear as shown in Figure~\ref{fig:catkin_init_build}.  
    Next, copy the \textit{two\_drones\_pkg} package into the \textit{src} folder:
    \begin{minted}[fontsize=\footnotesize,breaklines]{bash}
cp -a /source_path/two_drones_pkg ~/vnav_ws/src
    \end{minted}

    Then, build the workspace using:
    \begin{minted}[fontsize=\footnotesize,breaklines]{bash}
catkin build
    \end{minted}

    The build output is shown in Figure~\ref{fig:catkin_init_build}.  
    After the build completes, source the setup file to refresh the environment:
    \begin{minted}[fontsize=\footnotesize,breaklines]{bash}
source devel/setup.bash
    \end{minted}

    Finally, launch \textit{ROS} to visualize the static drone scene in \textit{RViz}:
    \begin{minted}[fontsize=\footnotesize,breaklines]{bash}
roslaunch two_drones_pkg two_drones.launch static:=True
    \end{minted}

    After adding the \textit{TF} display, three colored axes will appear for each coordinate frame, as illustrated in Figure~\ref{fig:two_drones_pkg}.  
    Remember to save the RViz configuration once the setup is complete.

    \item 
\end{enumerate}

\subsubsection{Observations}

\begin{figure}[H]
    \centering
    \begin{minipage}[b]{0.48\textwidth}
        \centering
        \includegraphics[width=\textwidth]{./source/img/partIV_1.png}
        \caption*{(a) catkin init}
    \end{minipage}
    \hfill
    \begin{minipage}[b]{0.48\textwidth}
        \centering
        \includegraphics[width=\textwidth]{./source/img/partIV_2.png}
        \caption*{(b) catkin build}
    \end{minipage}
    \caption{catkin init \& build}
    \label{fig:catkin_init_build}
\end{figure}

\begin{figure}[H]
    \centering
    \includegraphics[width=0.6\textwidth]{./source/img/partIV_3.png}
    \caption{Two drones}
    \label{fig:two_drones_pkg}
\end{figure}

\subsubsection{Discussion}
% TODO: Write the discussion of Part III.

%----------------------------------------------------------------------------------------
\subsection{Part IV}
\subsubsection{Objective}
% TODO: Write the objective of Part IV.

\subsubsection{Methodology}
% TODO: Write the methodology of Part IV.

\subsubsection{Observations}
% TODO: Write the observations of Part IV.

\subsubsection{Discussion}
% TODO: Write the discussion of Part IV.

%----------------------------------------------------------------------------------------
\section{Reflection and Analysis}
% TODO: Write a reflection and analysis of the lab report.

%----------------------------------------------------------------------------------------
\section{Conclusion}
% TODO: Write the conclusion of the lab report.
\newpage

%----------------------------------------------------------------------------------------
% Source Code
%----------------------------------------------------------------------------------------
\pagestyle{empty}
\section{Source Code}
\label{section:source_code}



\end{document}